\documentclass[12pt, a4paper, openany]{book}
\usepackage{cmap}
\usepackage[utf8]{inputenc}
\usepackage[T2A]{fontenc}
\usepackage[english,russian]{babel}
\usepackage{gensymb}
\usepackage[left=25mm, top=20mm, right=15mm, bottom=25mm, nohead, footskip=10mm]{geometry}

\usepackage{setspace,amssymb,amsfonts,amsmath,mathtext,cite,enumerate,float}

\usepackage[pdftex]{graphicx}
\usepackage{enumerate}
\graphicspath{{Images/}}
\DeclareGraphicsExtensions{.JPEG}
\usepackage[format=plain,labelsep=endash, font=small]{caption}

\usepackage{fancyhdr}
\usepackage{mathtools}
\usepackage{hyphenat}
\pagestyle{fancy}
\fancyhf{}
\fancyfoot[C]{\thepage}
\renewcommand{\headrulewidth}{0pt}

\renewcommand{\theenumi}{\arabic{enumi}}
\renewcommand{\theenumii}{\arabic{enumii}}
\renewcommand{\theenumiii}{\arabic{enumiii}}
\renewcommand{\labelenumi}{\arabic{enumi}.}
\renewcommand{\labelenumii}{\arabic{enumi}.\arabic{enumii}.}
\renewcommand{\labelenumiii}{\arabic{enumi}.\arabic{enumii}.\arabic{enumiii}.}
\setlength{\parindent}{1cm}
\renewcommand{\thesection}{\normalsize \arabic{section}.}
\renewcommand{\thesubsection}{\normalsize \arabic{section}.\normalsize \arabic{subsection}.}
\renewcommand{\thesubsubsection}{\normalsize \arabic{section}.\normalsize \arabic{subsection}.\normalsize \arabic{subsubsection}.}
\linespread{1.3}
    
\usepackage[hidelinks]{hyperref}

\usepackage{titlesec}

\usepackage{xcolor}
\usepackage{cancel}

\titleformat{\chapter}[display]
  {\centering\bfseries\Huge}
  {\chaptertitlename\ \thechapter}
  {20pt}
  {}

\titleformat{name=\section,numberless}[display]
  {\centering\bfseries}
  {}
  {0pt}
  {}

\usepackage{amsthm}
\usepackage{titlesec}
\usepackage{titletoc}

\usepackage{listings}

\usepackage{amssymb} 
\renewcommand\qedsymbol{\ensuremath{\blacksquare}}

\usepackage{amsmath}
\usepackage{tikz}
\usepackage{array}

\newtheorem*{definition}{Определение}
\newtheorem*{theorem}{Теорема}
\newtheorem*{statement}{Утверждение}
\title{Алгоритмы теории чисел}
\author{Рыжичкин Кирилл}
\date{Февраль 2024}

\begin{document}
\maketitle
\lstset{language=C++}

\tableofcontents

\newpage
\section*{Введение}
\addcontentsline{toc}{section}{Введение}{}

    Мир алгоритмов теории чисел представляет собой захватывающий путь в глубины математического мира, где числа становятся не только абстрактными сущностями, но и основой для разработки эффективных алгоритмов. Теория чисел, одна из старейших областей математики, занимается изучением свойств целых чисел и их взаимосвязей. Вместе с тем, алгоритмы теории чисел играют критическую роль в современном мире, находя применение в широком спектре областей, включая криптографию, компьютерные науки, теорию игр и многое другое.
    
    Мы начнем с основных понятий, таких как простые числа, делители, наибольший общий делитель и модульная арифметика, и постепенно углубимся в более сложные темы, такие как теорема Эйлера, китайская теорема об остатках, алгоритмы факторизации и многое другое.

% ГЛАВА 1
\chapter[Простейшие алгоритмы теории чисел]{Простейшие алгоритмы теории чисел}
\section{Простые числа}

\begin{definition}
    Натуральное число p называется простым (англ. prime number), если p>1 и p не имеет натуральных делителей, отличных от 1 и p.
\end{definition}

\begin{definition}
    Натуральное число n > 1 называется составным (англ. composite number), если n имеет по крайней мере один натуральный делитель, отличный от 1 и n.
\end{definition}

\noindent
    Согласно определениям, множество натуральных чисел разбивается на 3 подмножества:

\begin{enumerate}
    \item Простые числа.
    \item Составные числа.
    \item Число 1, которое не причисляется ни к простым, ни к составным числам.
\end{enumerate}

    Отсюда сразу возникает идея реализации интуитивно понятного алгоритма проверки числа n на простоту: если n = 1, вернуть false, иначе пройтись по всем числам от 2 до n - 1, и если ни на одно из них число не поделится, то вернуть true. Соответственно сложность такого алгоритма будет $O({n})$.

\newpage
\noindent
    Посмотрим на реализацию такого простейшего алгоритма:

\begin{lstlisting}
    bool isPrime(int n) {
        if (n == 1) {
            return false;
        }
        for (int i = 2; i < n; i++) {
            if (n % i == 0) {
                return false;
            }
        }
        return true;
    }
\end{lstlisting}

\begin{statement}
    Пусть $N=a \times b$, причем $a \leq b$. Тогда $a \leq \sqrt{N} \leq \boldsymbol{b}$
\end{statement}

\begin{proof}
    Eсли $a \leq b<\sqrt{N}$, то $a b \leq b^2<N$, но $a b=N$. А если $\sqrt{N}<a \leq b$, то $N<a^2 \leq a b$, но $a b=N$.
\end{proof}

    Из этого следует, что если число $N$ не делится ни на одно из чисел $2,3,4, \ldots,\lfloor\sqrt{N}\rfloor$, то оно не делится и ни на одно из чисел $\lceil\sqrt{N}\rceil+1, \ldots, N-2, N-1$, так как если есть делитель больше корня (не равный $N$), то есть делитель и меньше корня (не равный 1). Поэтому в цикле for достаточно проверять числа не до $N$, а до корня. Соответственно сложность такого алгоритма будет уже $O(\sqrt{n})$:

\begin{lstlisting}
    bool isPrime(int n) {
        if (n == 1) {
            return false;
        }
        for (int i = 2; i * i <= n; i++) {
            if (n % i == 0) {
                return false;
            }
        }
        return true;
    }
\end{lstlisting}

    Однако и этот алгоритм вовсе не оптимален и на практике применяется разве что на маленьких числах. Более оптимальные алгоритмы проверки на простоту будут расмотрены в следующих главах.

\newpage
\section{НОД и НОК двух чисел}

\begin{definition}
    Наибольший общий делитель (НОД) двух натуральных чисел \(a\) и \(b\) это наибольшее натуральное число \(d\), которое делит оба числа \(a\) и \(b\) без остатка. Обозначается как \(d = \text{НОД}(a, b)\).
\end{definition}

\begin{statement}
    Для любых двух натуральных чисел $a > b$ верно следующее равенство:
    \begin{equation*}
    \text{НОД}(a, b) = \text{НОД}(a - b, b).
    \end{equation*}
\end{statement}

\begin{proof}
    $n=$ \text{НОД}$(a, b) - $ наибольший среди всех общих делителей чисел $a$ и $b$, тогда $a$ $\vdots$ $n$ и $b$ $\vdots$ $n$.
    Отсюда $a-b$ $\vdots$ $n$, значит, $n$ - делитель $a-b$ и $b$, он не превосходит наибольшего из всех общих делителей чисел $a-b$ и $b$, т.е. $n \leqslant m=$ $\text{НОД}$(a, b). Аналогично, рассуждая получаем, что $a-b$ $\vdots$ $m$ и $b$ $\vdots$ $m=$ $\text{НОД}$(a, b). Отсюда $a=(a-b)+b$ $\vdots$ $m$ значит, $m$ - делитель $a$ и $b$, он не превосходит наибольшего и всех общих делителей чисел $a$ и $b$, т.е. $m \leqslant n=$ \text{НОД}$(a, b)$. Таким образом получаем $m \leqslant n \leqslant m$, т.е. \text{НОД}$(a, b)=n=m=$ \text{НОД}$(a-b, b)$.
\end{proof}

\noindent
\subsection{Алгоритм Евклида}

\begin{enumerate}
  \item \textbf{Инициализация}: Начнем с двух целых чисел $a$ и $b$, где $a \geq b$.
  
  \item \textbf{Шаг 1}: Разделим $a$ на $b$ и найдем остаток от деления. Пусть остаток обозначается как $r$. То есть, $a = bq_1 + r$, где $q_1$ - частное, а $r$ - остаток.
  
  \item \textbf{Шаг 2}: Если остаток $r$ равен нулю, то \text{НОД}$(a-b, b)$ равен $b$, и алгоритм завершается.
  
  \item \textbf{Шаг 3}: Если остаток $r$ не равен нулю, заменим $a$ на $b$ и $b$ на $r$, и вернемся к \textbf{шагу 1}.
  
  \item \textbf{Шаг 4}: Повторяем процесс, пока остаток $r$ не станет равным нулю. Когда это произойдет, \text{НОД}$(a, b)$ будет равен последнему ненулевому остатку, который был получен на предыдущем шаге.
  
  \item \textbf{Вывод результата}: Когда остаток становится равным нулю, последнее значение $b$ будет искомым \text{НОД}$(a, b)$.
\end{enumerate}

    Алгоритм Евклида гарантирует, что за конечное число шагов мы получим \text{НОД}$(a, b)$, так как на каждом шаге остаток уменьшается и стремится к нулю.

\newpage

\noindent
    Реализация алгоритма на C++:

\begin{lstlisting}
    int64_t NumberTheory::Gcd(int64_t a, int64_t b) {
      while (b != 0) {
        int64_t temp = b;
        b = a % b;
        a = temp;
      }
      return a;
    }
\end{lstlisting}

\begin{definition}
    Наименьшее общее кратное (НОК) двух натуральных чисел \(a\) и \(b\) это наименьшее натуральное число \(l\), которое делится на оба числа \(a\) и \(b\) без остатка. Обозначается как \(l = \text{НОК}(a, b)\).
\end{definition}

\begin{statement}
    $\forall a, b \in \mathbb{N}$:
    $\text{НОД}(a, b) \cdot \text{НОК}(a, b) = ab.$
\end{statement}

\begin{proof}
    \noindent
    Во время доказательства нам пригодится следующее простое свойство:
    $$
    \min (n ; m)+\max (n ; m)=n+m .
    $$
    \noindent
    Оно верно, т.к. $\min (n, m)$ совпадает с одним из чисел $n$ или $m$, $\operatorname{a }$ $\max (n, m)$ с другим.
    Запишем каноническое разложение на простые множители чисел $a$ и $b$ :
    $$
    a=p_1^{n_1} p_2^{n_2} \ldots p_k^{n_k} \text { и } b=p_1^{m_1} p_2^{m_2} \ldots p_k^{m_k}
    $$
    (возможно, что степени некоторых простых нулевые, т.е. простое число входит в разложение одного из чисел $a$ или $b$ на простые множители, но не входит в разложение второго). Запишем теперь разложение НОД $(a, b)$ и НОК $(a, b)$ :
    $$
    \begin{aligned}
    & \text {НОД}(a, b)=p_1^{\min \left\{n_1, m_1\right\}} p_2^{\min \left\{n_2, m_2\right\}} \ldots p_k^{\min \left\{n_k, m_k\right\}} \\
    & \text {НОК}(a, b)=p_1^{\max \left\{n_1, m_1\right\}} p_2^{\max \left\{n_2, m_2\right\}} \ldots p_k^{\max \left\{n_k, m_k\right\}}
    \end{aligned}
    $$
    \noindent
    То есть
    $$
    \begin{gathered}
    \text {НОД}(a, b) \cdot \operatorname{HOK}(a, b)=\prod_{i=1}^k p_i^{\min \left\{n_i, m_i\right\}+\max \left\{n_i, m_i\right\}}=
    \end{gathered}
    $$
    \noindent
    в силу свойства, указанного в самом начале доказательства, получаем
    $$
    =\prod_{i=1}^k p_i^{n_i + m_i}=\prod_{i=1}^k p_i^{n_i}\prod_{i=1}^k p_i^{m_i}=a b .
    $$
\end{proof}

\newpage
\noindent
    Отсюда алгоритм нахождения получается элементарным:
\begin{lstlisting}
    int64_t NumberTheory::Lcm(int64_t a, int64_t b) {
      return a * b / Gcd(a, b);
    }
\end{lstlisting}

\section{НОД и НОК n чисел}

    Алгоритм Евклида для нескольких чисел использует основное свойство наибольшего общего делителя: если $d$ делит каждое из чисел $a_1, a_2, …, a_n$, то $d$ также делит и их НОД.
    
\vspace{12pt}\noindent
    Ход алгоритма Евклида для n чисел:

\begin{enumerate}
    \item \textbf{Инициализация:} Пусть у нас есть набор чисел \(a_1, a_2, \ldots, a_n\).
    \item \textbf{Нахождение НОДа для первых двух чисел:}
    \[ d_1 = \text{НОД}(a_1, a_2) \]
    \item \textbf{Обновление НОДа для следующих чисел:} \\
    Для каждого \(i = 3, 4, \ldots, n\):
    \[ d_i = \text{НОД}(d_{i-1}, a_i) \]
    \item \textbf{Конечный результат:} \\
    Наибольший общий делитель всех чисел: НОД\((a_1, a_2, \ldots, a_n) = d_n\).
\end{enumerate}

    Поскольку $d_1$ является НОДом $a_1$ и $a_2$, он является делителем обоих этих чисел. Следовательно, $d_1$ делит оба числа нацело.
    
    Когда мы переходим к следующему числу $a_i$, мы находим НОД между текущим НОДом $d_{i-1}$ и $a_i$. Это значит, что $d_{i-1}$ делит $a_i$ и $d_{i-1}$ нацело. Поскольку $d_{i-1}$ также является НОДом всех предыдущих чисел, он делит все предыдущие числа нацело. Таким образом, $d{i-1}$ делит $a_i$ и все предыдущие числа нацело.
    
    Стало быть, после обработки всех чисел мы получаем $d_n$, который является НОДом всех чисел $a_1, a_2, …,a_n$.

\newpage
\noindent
    Реализация данного алгоритма на C++:

\begin{lstlisting}
    int64_t NumberTheory::Gcd(std::vector<int64_t> numbers) {
      int64_t res = numbers[0];
      for (int i = 1; i < numbers.size(); ++i) {
        res = Gcd(res, numbers[i]);
      }
      return res;
    }
\end{lstlisting}

\noindent
    Абсолютно аналогично получается алгоритм для нахождения НОКа нескольких чисел:

\begin{lstlisting}
    int64_t NumberTheory::Lcm(std::vector<int64_t> numbers) {
      int64_t res = numbers[0];
      for (int i = 1; i < numbers.size(); ++i) {
        res = Lcm(res, numbers[i]);
      }
      return res;
    }
\end{lstlisting}

\section{Расширенный алгоритм Евклида}

\begin{statement}
    $\forall a, b \in \mathbb{Z} \mid \text{НОД}(a, b) = d, \exists x, y \in \mathbb{Z}: ax + by = d$.
\end{statement}

    Основная идея расширенного алгоритма Евклида заключается в том, что мы можем выразить НОД двух чисел $a$ и $b$ через их линейную комбинацию, то есть такие целые числа $x$ и $y$, что $ax + by = \text{НОД}(a, b)$. Это следует из того факта, что множество всех линейных комбинаций $a$ и $b$ образует идеал в кольце $\mathbb{Z}$, а значит, содержит их НОД.
    
    Будем считать, что у нас есть структура ExtendedEuclideanResult, которая хранит НОД(a, b), а также коэффициенты x, y, тогда расширенный алгоритм Евклида на C++ будет выглядеть так:

\newpage
\begin{lstlisting}[breaklines=true]
    struct ExtendedEuclideanResult {
      ExtendedEuclideanResult(int64_t gcd, int64_t x, int64_t y) : gcd(gcd), x(x), y(y) {}
      ~ExtendedEuclideanResult() = default;
    
      int64_t gcd;
      int64_t x;
      int64_t y;
    
      friend std::ostream& operator<<(std::ostream& os, const ExtendedEuclideanResult& result) {
        os << "GCD:" << result.gcd << ",x:" << result.x << ",y:" << result.y;
        return os;
      }
    };
    
    ExtendedEuclideanResult NumberTheory::ExtEuclide(int64_t a, int64_t b) {
      int64_t x = 0, y = 1, lastX = 1, lastY = 0, temp;
      while (b != 0) {
        int64_t quotient = a / b;
        int64_t remainder = a % b;
    
        a = b;
        b = remainder;
    
        temp = x;
        x = lastX - quotient * x;
        lastX = temp;
    
        temp = y;
        y = lastY - quotient * y;
        lastY = temp;
      }
    
      return ExtendedEuclideanResult(a, lastX, lastY);
    }
\end{lstlisting}

\section{Линейные диофантовы уравнения с двумя
неизвестными}

    Пусть нам дано уравнение $ax + by = c$, где $a, b, c \in \mathbb{Z}$.
    
    Заметим, что левая часть уравнения делится на $\text{НОД}(a, b)$, значит $c$ обязано делиться на $\text{НОД}(a, b)$, чтобы решение существовало.
    
    Если $\text{НОД}(a, b) \mid c$, то поделим обе части на этот НОД, получим новое уравнение $a^{'}x + b^{'}y = c^{'}$, где $\text{НОД}(a^{'}, b^{'}) = 1$.
    
    Пусть мы угадали какое-то решение $(x_0, y_0)$ этого уравнения. Так как $a^{'}x_0 + b^{'}y_0 = c^{'}$, то для любой пары $(x, y)$ получим $a^{'}x + b^{'}y = a^{'}x_0 + b^{'}y_0$. Получим $a^{'}(x - x_0) = b^{'}(y-y_0)$.
    
    Так как $\text{НОД}(a^{'}, b^{'}) = 1$, то $b \mid x - x_0$, обозначим $x - x_0 = kb^{'}$, тогда $y - y_0 = ka^{'}$.

\noindent
    В итоге получаем множество решений $(x, y) = (x_0 + kb^{'}, y_0 - ka^{'})$, где $k \in \mathbb{Z}$.
    
    Угадать решение уравнения $a^{'}x + b^{'}y = c^{'}$, где $\text{НОД}(a^{'}, b^{'}) = 1$ можно с помощью алгоритма Евклида. Сначала найдем решение $(x_0, y_0)$ уравнения $a^{'}x + b^{'}y = 1$ (это можно сделать в силу утверждения со стр. 9), тогда $(cx_0, cy_0)$ - решение уравнения $a^{'}x + b^{'}y = c^{'}$.

\subsection{Алгоритм решения линейного диофантова уравнения с двумя неизвестными}

\begin{enumerate}
    \item \textbf{Проверка существования решений:} Проверим, делится ли правая часть уравнения на $\text{НОД}(a, b)$, если да - идём дальше, если нет - решения отсутствуют.
    \item \textbf{Нахождение одного частного решения:} Применяя алгоритм Евклида, находим частное решение $(x_0, y_0)$ уравнения $ax + by = c$.
    \item \textbf{Нахождение общего решения:} Общее решение линейного диофантова уравнения может быть представлено в виде \\
    \[(x_0 + \frac{b}{\text{НОД}(a, b)} t, y_0 - \frac{a}{\text{НОД}(a, b)} t), t \in \mathbb{Z}\]
\end{enumerate}

\newpage
\noindent
    Так выглядит реализация данного алгоритма на C++:

\begin{lstlisting}[breaklines=true]
    struct DiophantusResult {
      DiophantusResult() = default;
      DiophantusResult(int64_t x, int64_t y, int64_t k_1, int64_t k_2) : x(x), y(y), k_1(k_1), k_2(k_2), s("good") {}
      ~DiophantusResult() = default;
    
      int64_t x = 0;
      int64_t y = 0;
      int64_t k_1 = 0;
      int64_t k_2 = 0;
      std::string s = "none";
    
      friend std::ostream& operator<<(std::ostream& os, const DiophantusResult& result) {
        std::string sign_1 = "+";;
        std::string sign_2 = "-";
    
        if (result.k_1 < 0) {
          sign_1 = "-";
        }
    
        if (result.k_2 < 0) {
          sign_2 = "+";
        }
    
        if (result.s == "good") {
          os << "(x,y)=" << "(" << result.x << sign_1 << std::abs(result.k_1) << "t," << result.y << sign_2 << std::abs(result.k_2) << "t)";
        }
    
        else {
          os << "None";
        }
    
        return os;
      }
    };
    
    DiophantusResult NumberTheory::Diophantus(int64_t a, int64_t b, int64_t c) {
      ExtendedEuclideanResult result = ExtEuclide(a, b);
    
      int64_t g = result.gcd;
    
      if (c % g != 0) {
        return DiophantusResult(); 
      }
    
      int64_t k = c / g;
    
      return DiophantusResult(k * result.x, k * result.y, b / g, a / g);
    }
\end{lstlisting}

\section{Количество/сумма делителей числа}

    Алгоритм нахождения количества/суммы делителей числа можно реализовать элементарным перебором делителей. Сложность такого алгоритма будет $\text{O}(\sqrt{n})$. 

\noindent
    Рассмотрим такие алгоритмы:

\begin{lstlisting}[breaklines=true]
    int64_t NumberTheory::DivisorsCount(int64_t n) {
      n = std::abs(n);
      int64_t count = 0;
      for (int64_t i = 1; i <= static_cast<int64_t>(std::sqrt(n)); ++i) {
        if (n % i == 0) {
          count += (i == n / i) ? 1 : 2;
        }
      }
      return count;
    }
\end{lstlisting}

\newpage
\begin{lstlisting}[breaklines=true]
    int64_t NumberTheory::DivisorsSum(int64_t n) {
      for (int64_t i = 1; i <= static_cast<int64_t>(std::sqrt(n)); ++i) {
        if (n % i == 0) {
          sum += i;
          if (i != n / i) {
            sum += n / i;
          }
        }
      }
      return sum;
    }
\end{lstlisting}

    Однако количество/сумма делителей любого числа легко выражается через его разложение на простые множители. Соответственно если мы реализуем факторизацию со сложностью, меньшей чем $\text{O}(\sqrt{n})$, то получим более оптимальный алгоритм. Такие факторизации будут рассмотрены в следующих главах, а пока будем считать, что у нас есть такой алгоритм разложения на простые множители и получим формулы количества и суммы делителей числа.

\noindent
    Пусть дано целое число $ n $ с разложением на простые множители:
    \[n = p_1^{a_1} \times p_2^{a_2} \times \ldots \times p_k^{a_k},\]
\noindent
    где $ p_1, p_2, \ldots, p_k $ - простые числа (не обязательно различные), а $ a_1, a_2, \ldots, a_k $ - их показатели степени.
    
\begin{statement}
    Пусть $\tau(n)$ обозначает количество положительных делителей числа $n$. Тогда $\tau(n) = (a_1 + 1) \times (a_2 + 1) \times \ldots \times (a_k + 1)$.
    \end{statement}
    
    \begin{proof}
    Каждый делитель числа $ n $ имеет вид:
    \[ d = p_1^{b_1} \times p_2^{b_2} \times \ldots \times p_k^{b_k},\]
    \noindent
    где $ 0 \leq b_i \leq a_i $ для $ i = 1, 2, \ldots, k $. Чтобы получить все делители числа $ n $, мы можем взять каждый из $ k $ простых множителей и выбрать любую комбинацию показателей степени от 0 до $ a_i $. Таким образом, общее количество делителей числа $ n $:
    \begin{equation}\tau(n) = (a_1 + 1) \times (a_2 + 1) \times \ldots \times (a_k + 1)\end{equation}.
\end{proof}

\newpage

\begin{statement}
    Пусть $\sigma(n)$ обозначает сумму положительных делителей числа $n$. Тогда \begin{equation}
    \sigma(n) = \frac{{p_1^{a_1+1} - 1}}{{p_1 - 1}} \cdot \frac{{p_2^{a_2+1} - 1}}{{p_2 - 1}} \cdot \ldots \cdot \frac{{p_k^{a_k+1} - 1}}{{p_k - 1}}
    \end{equation}
\end{statement}

\begin{proof}
    Сумма делителей числа \( n \) может быть выражена как произведение всех возможных комбинаций делителей:
    \[
    \sigma(n) = (1 + p_1 + p_1^2 + \ldots + p_1^{a_1}) \cdot (1 + p_2 + p_2^2 + \ldots + p_2^{a_2}) \cdot \ldots \cdot (1 + p_k + p_k^2 + \ldots + p_k^{a_k})
    \]
    \noindent
    Сумма геометрической прогрессии имеет вид:
    \[
    1 + p^1 + p^2 + \ldots + p^m = \frac{{p^{m+1} - 1}}{{p - 1}}
    \]
    \noindent
    Подставляя это обратно в наше уравнение, получаем требуемое равенство.
\end{proof}

\section{Степень вхождения простого числа в факториал}

\begin{statement}
    Пусть $\text{ord}_p(n!)$ обозначает степень вхождения простого числа $p$ в $n!$. Тогда
    \begin{equation}\text{ord}_p(n!) = \sum_{i=1}^{\infty} \left\lfloor \frac{n}{p^i} \right\rfloor\end{equation}
\end{statement}

\begin{proof}
    \[n! = 1 \times 2 \times 3 \times \ldots \times (n-1) \times n\]
    
    Каждый $p$-ый член этого произведения делится на $p$, т.е. даёт +1 к ответу, количество таких членов равно $\left\lfloor \frac{n}{p} \right\rfloor.$
    
    Далее, заметим, что каждый $p^2$-ый член этого ряда делится на $p^2$, т.е. даёт ещё +1 к ответу (учитывая, что $p$ в первой степени уже было учтено до этого); количество таких членов равно $\left\lfloor \frac{n}{p^2} \right\rfloor$.
    
    И так далее, каждый $p^i$-ый член ряда даёт +1 к ответу, а количество таких членов равно $\left\lfloor \frac{n}{p^i} \right\rfloor.$
    
    Таким образом, \[\text{ord}_p(n!) = \left\lfloor\frac{n}{p}\right\rfloor + \left\lfloor\frac{n}{p^2}\right\rfloor + \ldots + \left\lfloor\frac{n}{p^i}\right\rfloor + \ldots\].
\end{proof}

\newpage

\noindent
    Реализация данного алгоритма на C++:
\begin{lstlisting}[breaklines=true]
    int64_t NumberTheory::PrimePowerInFactorial(int64_t n, int64_t p) {
    	int64_t res = 0;
    	while (n) {
    		n /= p;
    		res += n;
    	}
    	return res;
    }
\end{lstlisting}

\section{Решето Эратосфена}

    Решето Эратосфена --- достаточно эффективный алгоритм для нахождения всех простых чисел в отрезке от 1 до $n$ за $\text{O}(nloglogn)$. Ход алгоритма:

\begin{enumerate}
    \item Начинаем с списка чисел от 2 до $n$.
    \item Отмечаем первое простое число в списке (2) как простое.
    \item Зачеркиваем все кратные двойке числа в списке (кроме самой двойки).
    \item Переходим к следующему незачеркнутому числу в списке (3), отмечаем его как простое.
    \item Зачеркиваем все кратные тройке числа в списке (кроме самой тройки).
    \item Повторяем этот процесс для каждого незачеркнутого числа в списке, пока не достигнем $\sqrt{n}$.
    \item Все оставшиеся незачеркнутые числа в списке считаются простыми.
\end{enumerate}

\begin{table}[h]
\centering
\begin{tabular}{|c|c|c|c|c|c|c|c|c|c|}
    \hline
    \textcolor{gray}{} & 2 & 3 & \textcolor{gray}{4} & 5 & \textcolor{gray}{6} & 7 & \textcolor{gray}{8} & \textcolor{gray}{9} & \textcolor{gray}{10} \\
    \hline
    11 & \textcolor{gray}{12} & 13 & \textcolor{gray}{14} & \textcolor{gray}{15} & \textcolor{gray}{16} & 17 & \textcolor{gray}{18} & 19 & \textcolor{gray}{20} \\
    \hline
    \textcolor{gray}{21} & \textcolor{gray}{22} & 23 & \textcolor{gray}{24} & \textcolor{gray}{25} & \textcolor{gray}{26} & \textcolor{gray}{27} & \textcolor{gray}{28} & 29 & \textcolor{gray}{30} \\
    \hline
    31 & \textcolor{gray}{32} & \textcolor{gray}{33} & \textcolor{gray}{34} & \textcolor{gray}{35} & \textcolor{gray}{36} & 37 & \textcolor{gray}{38} & \textcolor{gray}{39} & \textcolor{gray}{40} \\
    \hline
    41 & \textcolor{gray}{42} & 43 & \textcolor{gray}{44} & \textcolor{gray}{45} & \textcolor{gray}{46} & 47 & \textcolor{gray}{48} & \textcolor{gray}{49} & \textcolor{gray}{50} \\
    \hline
    \textcolor{gray}{51} & \textcolor{gray}{52} & 53 & \textcolor{gray}{54} & \textcolor{gray}{55} & \textcolor{gray}{56} & \textcolor{gray}{57} & \textcolor{gray}{58} & 59 & \textcolor{gray}{60} \\
    \hline
    61 & \textcolor{gray}{62} & \textcolor{gray}{63} & \textcolor{gray}{64} & \textcolor{gray}{65} & \textcolor{gray}{66} & 67 & \textcolor{gray}{68} & \textcolor{gray}{69} & \textcolor{gray}{70} \\
    \hline
    71 & \textcolor{gray}{72} & 73 & \textcolor{gray}{74} & \textcolor{gray}{75} & \textcolor{gray}{76} & \textcolor{gray}{77} & \textcolor{gray}{78} & 79 & \textcolor{gray}{80} \\
    \hline
    \textcolor{gray}{81} & \textcolor{gray}{82} & 83 & \textcolor{gray}{84} & \textcolor{gray}{85} & \textcolor{gray}{86} & \textcolor{gray}{87} & \textcolor{gray}{88} & 89 & \textcolor{gray}{90} \\
    \hline
    \textcolor{gray}{91} & \textcolor{gray}{92} & \textcolor{gray}{93} & \textcolor{gray}{94} & \textcolor{gray}{95} & \textcolor{gray}{96} & 97 & \textcolor{gray}{98} & \textcolor{gray}{99} & \textcolor{gray}{100} \\
    \hline
    \end{tabular}
    \caption{Иллюстрация решета Эратосфена}
\end{table}

\newpage
\noindent
    Реализация данного алгоритма на C++:

\begin{lstlisting}[breaklines=true]
    std::vector<int64_t> NumberTheory::SieveOfEratosthenes(int64_t n) {
      std::vector<bool> isPrime(n + 1, true);
      std::vector<int64_t> primes;
    
      for (int p = 2; p * p <= n; p++) {
        if (isPrime[p] == true) {
          for (int64_t i = p * p; i <= n; i += p)
            isPrime[i] = false;
        }
      }
    
      for (int64_t p = 2; p <= n; p++) {
        if (isPrime[p])
          primes.push_back(p);
      }
    
      return primes;
    }
\end{lstlisting}

% ГЛАВА 1
\chapter[Модульная арифметика]{Модульная арифметика}
\section{Сравнения по модулю}

\begin{definition}
    Пусть $a$, $b$, и $m$ --- целые числа, где $m > 0$. Мы говорим, что $a$ сравнимо с $b$ по модулю $m$, если $m$ делит разность $a - b$, обозначается как \[a \equiv b \pmod{m}\]
\end{definition}

\subsection{Свойства сравнений}

\begin{enumerate}
    \item Если $a \equiv b \pmod{m}$ и $b \equiv c \pmod{m}$, то $a \equiv c \pmod{m}$.
    \item Если $a \equiv b \pmod{m}$, то $ak \equiv bk \pmod{m}$ для любого целого числа $k$.
    \item Если $a \equiv b \pmod{m}$ и $c \equiv d \pmod{m}$, то $a + c \equiv b + d \pmod{m}$.
    \item Если $a \equiv b \pmod{m}$ и $c \equiv d \pmod{m}$, то $ac \equiv bd \pmod{m}$.
    \item Если $a \equiv b \pmod{m}$, то $a^k \equiv b^k \pmod{m}$ для любого натурального числа $k$.
\end{enumerate}

\section{Обратный элемент в кольце вычетов по модулю}

\begin{definition}
    Пусть $a$ и $m$ - целые числа, причем $m > 1$. Целое число $a^{-1}$ называется обратным по модулю $m$ элементом к $a$, если выполняется условие:
    \[aa^{-1} \equiv 1 \pmod{m}\]
\end{definition}

\newpage
\subsection{Существование и единственность}

\begin{statement}
    Обратный по модулю $n$ элемент к $a$ существует тогда и только тогда, когда $a$ и $n$ взаимно просты.
\end{statement}

\begin{proof} 
    ($\Rightarrow$) Предположим, что существует обратный по модулю $n$ элемент к $a$. Тогда существует такое целое число $b$, что $ab \equiv 1 \pmod{n}$. Это означает, что существует такое целое число $k$, что $ab = 1 + kn$. Таким образом, $1 = ab - kn$, что означает, что $1$ является линейной комбинацией $a$ и $n$. Следовательно, $a$ и $n$ взаимно просты по определению.
    
    ($\Leftarrow$) Теперь предположим, что $a$ и $n$ взаимно просты. По расширенному алгоритму Евклида существуют такие целые числа $x$ и $y$, что $ax + ny = 1$. Заметим, что $ax \equiv 1 \pmod{n}$. Таким образом, $x$ является обратным по модулю $n$ элементом к $a$.
\end{proof}

\subsection{Нахождение обратного элемента}

    Пусть нам надо найти обратный к $a$ элемент по модулю $m$, то есть мы ищем $x$ такой, что \[ax \equiv 1 \pmod{m}\] 
    Рассмотрим линейное диофантово уравнение $at+my=1$, левая часть при делении на $m$ даёт остаток $at$, а правая --- $1$, то есть чтобы найти обратный к $a$ элемент, нам достаточно найти коэффициент $t$ с помощью расширенного алгоритма Евклида и взять его по модулю $m$.

\noindent
Реализация на C++:

\begin{lstlisting}[breaklines=true]
    int64_t NumberTheory::ModInverse(int64_t a, int64_t m) {
      // no solutions 
      if (a == 0 || Gcd(m, a) != 1) {
        return 0;
      }
    
      ExtendedEuclideanResult euclide = ExtEuclide(a, m);
      int64_t result = ((euclide.x % m) + m) % m;
    
      return result;
    }
\end{lstlisting}

\section{Линейные сравнения с неизвестной}

\begin{definition}
    Линейное сравнение по модулю $m$ представляет собой уравнение вида $ax \equiv b \pmod{m}$, где $a, b, m \in \mathbb{Z}, m > 0$, а $x$ - неизвестная переменная.
\end{definition}

    Обозначим $d = \text{НОД}(a, m)$. Так как $ax$ и $b$ дают одинаковые остатки по модулю $m$, то $b$ обязано делиться на $d$. Если $b$ не делится на $d$, то сравнение решений не имеет.
    Если же $b$ делится на $d$, то разделим обе части сравнения на $d$ и перейдём к сравнению 
    \begin{equation}a^{'}x \equiv b^{'} \pmod{m^{'}}\end{equation}
    Так как $\text{НОД}(a^{'}, m^{'}) = 1$, то существует элемент $a^{'-1}$, для которого верно
    \begin{equation}a^{'}a^{'-1} \equiv 1 \pmod{m^{'}}\end{equation}
    Умножим сравнение (2.1) на $a^{'-1}$, тогда
    \begin{equation}x \equiv a^{'-1}b^{'} \pmod{m^{'}}\end{equation}
    Соответственно решением исходного сравнения будут являться все $x$, дающие по модулю $m^{'}$ остаток $a^{'-1}b^{'}$, и находящиеся в кольце $\mathbb{Z}/m\mathbb{Z}$.
    Окончательно получаем:
\begin{equation}
    x \in \{a^{'-1}b^{'} + km^{'}\, |\, k \in \mathbb{Z},\, 0 <= k < d\}
\end{equation}

\noindent
    Реализация алгоритма решения линейного сравнения на C++:

\begin{lstlisting}[breaklines=true]
    std::vector<int64_t> NumberTheory::SolveLinearCongruence(int64_t a, int64_t b, int64_t m) {
      int64_t d = Gcd(a, m);
      if (b % d != 0)
        return std::vector<int64_t>{ 0 }; // no solutions
      a /= d;
      b /= d;
      m /= d;
      int64_t x_0 = ModInverse(a, m) * b % m;
      std::vector<int64_t> result{ x_0 };
      for (int64_t k = 1; k < d; ++k) {
        result.push_back(x_0 + k * m);
      }
      return result;
    }
\end{lstlisting}

\section{Китайская теорема об остатках (КТО)}

\begin{theorem}
    Китайская теорема об остатках утверждает, что для любых целых чисел $a_1, a_2, \ldots, a_n$ и модулей $m_1, m_2, \ldots, m_n$, если модули попарно взаимно простые (то есть, $\text{НОД}(m_i, m_j) = 1$ для всех $i \neq j$), то система сравнений:
    \[
    \begin{aligned}
    x &\equiv a_1 \pmod{m_1} \\
    x &\equiv a_2 \pmod{m_2} \\
    &\vdots \\
    x &\equiv a_n \pmod{m_n}
    \end{aligned}
    \]
    имеет ровно одно решение $x$, которое можно найти с помощью алгоритма КТО.
\end{theorem}

\subsection{Алгоритм поиска решений}
\begin{enumerate}
    \item \textbf{Подготовка данных:} Пусть дана система линейных сравнений:
    \[
    \begin{aligned}
    x &\equiv a_1 \pmod{m_1} \\
    x &\equiv a_2 \pmod{m_2} \\
    &\vdots \\
    x &\equiv a_n \pmod{m_n}
    \end{aligned}
    \]
    
    \item \textbf{Нахождение общего модуля:} Вычисляем общий модуль $M$, который равен произведению всех модулей $m_i$.
    
    \item \textbf{Вычисление $M_i$ и $m_i$:} Для каждого $i$ вычисляем $M_i = \frac{M}{m_i}$, и $n_i$ обратное $M_i$ по модулю $m_i$.
    
    \item \textbf{Нахождение $x$:} Для каждого $i$ вычисляем (изначально $x = 0$):
    \[
    x = (x + a_i \cdot M_i \cdot n_i) \mod M
    \]
    
    \item \textbf{Возврат результата:} Результат $x$ является искомым числом, которое удовлетворяет всей системе линейных сравнений.
\end{enumerate}

\newpage
\noindent
    Пример реализации алгоритма КТО на C++:
    
\begin{lstlisting}[breaklines=true]
    struct ChineseRemainderTheoremResult {
      ChineseRemainderTheoremResult(int64_t res, int64_t mod) : res(res), mod(mod) {}
      ~ChineseRemainderTheoremResult() = default;
    
      int64_t res = 0;
      int64_t mod = 0;
    
      friend std::ostream& operator<<(std::ostream& os, const ChineseRemainderTheoremResult& result) {
        os << "Solution: x = " << result.res << " (mod " << result.mod << ")";
        return os;
      }
    };
    
    ChineseRemainderTheoremResult NumberTheory::ChineseRemainderTheorem(const std::vector<int64_t>& r, const std::vector<int64_t>& m) {
      int64_t n = r.size();
      int64_t M = std::accumulate(m.begin(), m.end(), 1, std::multiplies<int64_t>());
    
      std::vector<int64_t> Mod(n);
      std::vector<int64_t> mod(n);
      for (int64_t i = 0; i < n; ++i) {
        Mod[i] = M / m[i];
        mod[i] = ModInverse(Mod[i], m[i]);
      }
    
      int64_t x = 0;
      for (int i = 0; i < n; ++i) {
        x = (x + r[i] * Mod[i] * mod[i]) % M;
      }
    
      return ChineseRemainderTheoremResult(x, M);
    }
\end{lstlisting}

\subsection{Пример работы алгоритма}

Для системы сравнений:
    \[
    \begin{aligned}
    x &\equiv 2 \pmod{3} \\
    x &\equiv 3 \pmod{5} \\
    x &\equiv 2 \pmod{7}
    \end{aligned}
    \]

\noindent
    Шаги алгоритма КТО следующие:

\begin{enumerate}
    \item Общий модуль $M = 3 \times 5 \times 7 = 105$.
    \item $M_1 = 105/3 = 35$, $M_2 = 105/5 = 21$, $M_3 = 105/7 = 15$.
    \item $m_1 = 35^{-1} \mod 3 = 2$, $m_2 = 21^{-1} \mod 5 = 1$, $m_3 = 15^{-1} \mod 7 = 1$.
    \item $x = (2 \cdot 35 \cdot 2 + 3 \cdot 21 \cdot 1 + 2 \cdot 15 \cdot 1) \mod 105 = 23$.
\end{enumerate}

\noindent
    Таким образом, решением системы является $x \equiv 23 \pmod{105}$.

\end{document}

